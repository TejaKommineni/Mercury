\section{Background}

The mobile network ecosystem is complex, and host to a number of
related efforts in providing context-specific messages between
endpoints in an Intelligent Transportation System environment. Before
we discuss the design and implementation of \name, we couch it in
the context of prior work. In addition, we provide backround on the
3GPP 4G Evolved Packet System~\cite{4G}, which is an important and
widely used mobile networking ecosystem that we position \name{} 
within.

%
% RELATED WORK
%
\subsection{Related Work}

There has been plenty of attention paid to effecient and reliable
delivery of messages within VANETs.  Much of this focuses on multi-hop
clustering and hybrid use of evolved packet system RAN (LTE).  The
VMaSC~\cite{ucar2016multihop}, MDMAC~\cite{wolny2008modified}, and
NHop~\cite{zhang2011novel} systems attempt to form stable mobile
802.11p clusters, using the LTE network to bridge between disconnected
clusters. These solutions largely ignore the details of the mobile
core network, glossing over questions of component placement and the
resulting effects on latency. Moreover, these systems are
complimentary to \name{} in that they can be used to reduce LTE
resource contention and improve reliable transfer of messages. Such
integration would however come at the cost of additional complexity,
failure modes, and latency due to additional network path segments.

From the publish-subscribe perspective, there is a large volume of
prior work on traditional pubsub mechanisms~\cite{hartenstein2008tutorial}
\cite{mir2014lte} \cite{kim2012performance} \cite{araniti2013lte}.
This work is complementary to ours since it focuses on the useful
aspects of pubsub, which we largely wish to reuse. There has also been
work on pubsub systems focused on mobile endpoints. The
MoPS~\cite{nasim2014mobile} publish-subscribe system scales
efficiently for large numbers of clients and deals well with changing
broker association.  \name{} could replace the \pubsub system used in
the prototype with MoPS to better target mobile endpoints. However,
MoPS does not include the area of interest concept, which would
continue to be handled by the \name{} broker. A paper by
Pongthawornkamol et al~\cite{pongthawornkamol2007analysis} looks at
pubsub in the context of ad hoc wireless networks. However, this work
only performs simulations of mechanisms, and does not consider a
larger top-down vantage point (important for coordination in a large
ITS).

Location-aware messaging, or geo-routing, in vehicular networks has
been studied fairly extensively~\cite{bilal2013position}.
Nevertheless, we find that most work has only proposed and simulated
mechanisms. Furthermore, many studies look at ad hoc networks and
fine-grained positioning of endpoints within vehicle clusters.  While
such mechanisms may be helpful for real-time collision avoidance, we
argue that they tend to be overly complex and unnecessarily constrain
the communication domain to clusters of endpoints versus a central
system with a global view. The argument against centralized systems is
frequently that mobile networks are overloaded. This may be true in
instances of particularly high device concentration (e.g. music
festivals), but we found no studies showing a general lack of
available RAN resources outside of such crowded contexts.
Furthermore, in an ITS environment, vehicle populations and
anticipated growth could be used to properly size capacity.

The related work outline above focuses on particular aspects of
messaging.  To the best of our knowledge, no messaging system targeted
at mobile endpoints simultaneously incorporates aspects we consider
crucial to a holistic messaging platform for Intelligent
Transportation Systems.  Such a system should enable a global view of
endpoints to facilitate coordinated decision-making with complete
data. It should take advantage of mobile network environment
mechanisms (e.g. eMBMS~\cite{lecompte2012evolved}) to reduce overhead and 
latency. It should
consider the placement of components within the mobile network and the
impact of this placement.  Finally, an ITS messaging system should
provide a location-aware addressing mechanism. The \name{} messaging
system is designed with all of these aspects in mind.


\subsection{Mobile Networking Ecosystem}

\name{} is primarily framed in the context of the 3GPP 4G and emerging
5G mobile networking architectures. Therefore, we provide some
background on these systems.  The vast majority of mobile carriers
utilize these Evolved Packet Systems (EPS), making them an especially
relevant environment in which to operate a mobile messaging system.
Note that \name{} is also amenable to other mobility-friendly network
architectures, such as MobilityFirst~\cite{raychaudhuri2012mobilityfirst}, 
but we
focus the discussion in this paper on the 3GPP EPS.  The 4G system has
been in active deployment since 2008~\cite{chen2015financial}, and has
undergone a number of revisions. Also in play in some deployment
scenarios (see section~\ref{sec:deployments}) are software defined
infrastructure concepts that are expected to be prominent components
in the upcoming 5G EPS~\cite{5gvision}.

The 4G EPS includes the following key service functions relevant to
\name: Mobility Managment Entity (MME), Home Subscriber Service
(HSS), Serving Gateway (SGW), Packet Data Network Gateway (PDN-GW or
more commonly PGW), evolved NodeB (eNodeB), and User Equipment (UE).
We will briefly describe the role of each of these components, and
their relationships with one another and with \name. The \name{} 
architecture diagram in figure~\ref{fig:arch} shows \name{} 
components in the context of a 4G EPS.

We will briefly cover the 4G components next. \textbf{User Equipment
  (UE)} typically refers to end user devices such as mobile phones,
tablets, and 4G radio equipped laptops. The \name{} Endpoint component
runs on these. The \textbf{Mobility Management Entity (MME)} is the 4G
control plane function responsible for tracking the live (dynamic)
session state for UEs. The \textbf{Home Subscriber Service (HSS)} is
essentially a database of user (subscriber) information. The
\textbf{Serving Gateway (SGW)} is the first data tunnel anchor point
that UE sessions connect through (GTP tunnels). The \textbf{Packet
  Data Network Gateway (PGW)} act as the egress point for a large
number of UE data bearers (GTP tunnels); they fan out to multiple
SGWs.  \textbf{Evolved NodeB (eNodeB)} devices are the wireless access
points of the 4G EPS. They bridge the radio access network (RAN)
through which the mobile endpoints (UEs) directly communicate with the
evolved packet core (EPC). GTP tunnels are established for each UE
between the eNodeB it is associated with and an upstream SGW.  The
eNodeB also initiates session setup and default data bearer
establishment when UEs attach, acting as a proxy for UE to MME
control plane signalling. eNodeBs covering adjacent cells coordinate
through the MME and possibly with one another to accomplish handover
as endpoints move.

\subsection{The Vehicle Environment}

The environment of a vehicle endpoint is highly relevant context for
drivers and passengers. We argue that traffic, safety issues, and
proximity to resources are key to decision making in this
setting. Therefore, related telemetry such as position, speed, and
direction should be relayed for analysis. Vehicle \emph{position} is
almost universally measured using GPS~\cite{misra2006global}.
Although known radio access points can be used for rough localization,
such positions are often too coarse-grained.  In fact, a better system
would make use of computer vision and/or static environmental sensors
to extract lane position, relative distances to other vehicles, etc.
Our work uses GPS coordinates and leaves more fine-grained placement
techniques to future work. \emph{Speed} is to be collected by vehicle
speedometers, or through monitoring of accelerometers such as are
found in most smart phones. Finally, to acquire \emph{direction},
digital compass readings can be gathered from smart phones and are
also available in many onboard vehicle systems.  We expect that
environmental sensors will grow in sophistication and accuracy through
the proliferation of self-driving cars and ITS.
