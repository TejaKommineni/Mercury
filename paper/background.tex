\section{Background}

The mobile network ecosystem is complex, and host to a number of
related efforts in providing context-specific messages between
endpoints in an Intelligent Transportation System environment. Before
we discuss the design and implementation of Mercury, we discuss it in
the context of prior work. In addition, we provide backround on the
3GPP 4G Evolved Packet System~\cite{4G}, which is an important and
widely used mobile networking ecosystem that we position Mercury
within.

%
% RELATED WORK
%

There has been plenty of attention paid to effecient and reliable
delivery of messages within VANETs.  Much of this focuses on multi-hop
clustering and hybrid use of evolved packet system RAN (LTE).  The
VMaSC~\cite{VMaSC}, MDMAC~\cite{MDMAC}, and NHop~\cite{NHop} systems
attempt to form stable mobile 802.11p clusters, using the LTE network
to bridge between disconnected clusters. {\bf INSERT INFO ON CMGM.}
These systems are complimentary to Mercury in that they can be used to
reduce LTE resource contention and improve reliable transfer of
messages.

{\bf NEED MORE PUBSUB CITATIONS.}
The MoPS~\cite{MoPS} publish-subscribe system scales efficiently for
large numbers of clients and deals well with changing broker
association.  Mercury could replace the Emulab pubsub system used in
the prototype with MoPS to help it scale better. MoPS does not include
the area of interest concept, which would continue to be handled by
the Mercury broker.


\subsection{Mobile Networking Ecosystem}

Mercury is primarily framed in the context of the 3GPP 4G and emerging
5G mobile networking architectures. Therefore, we provide some
background on these systems.  The vast majority of mobile carriers
utilize these Evolved Packet Systems (EPS)~\cite{mobile-stats}, making
them an especially relevant environment in which to operate a mobile
messaging system.  Note that Mercury is also amenable to other
mobility-friendly network architectures, such as
MobilityFirst~\cite{mobility-first}, but we focus the discussion in
this paper on the 3GPP EPS.  The 4G system has been in active
deployment since 2008~\cite{chen2015financial}, and has undergone a
number of revisions. Also in play in some deployment scenarios (see
section~\ref{sec:deployments}) are software defined infrastructure
concepts that are expected to be prominent components in the upcoming
5G EPS~\ref{5gvision}.

The 4G EPS includes the following key service functions relevant to
Mercury: Mobility Managment Entity (MME), Home Subscriber Service
(HSS), Serving Gateway (SGW), Packet Data Network Gateway (PDN-GW or
more commonly PGW), evolved NodeB (eNodeB), and User Equipment (UE).
We will briefly describe the role of each of these components, and
their relationships with one another and with Mercury. The Mercury
architecture diagram in figure~\ref{fig:arch} shows Mercury
components in the context of a 4G EPS.

We will briefly cover the 4G components next. \textbf{User Equipment
  (UE)} typically refers to end user devices such as mobile phones,
tablets, and 4G radio equipped laptops. The Mercury Endpoint component
runs on these. The \textbf{Mobility Management Entity (MME)} is the 4G
control plane function responsible for tracking the live (dynamic)
session state for UEs. The \textbf{Home Subscriber Service (HSS)} is
essentially a database of user (subscriber) information. The
\textbf{Serving Gateway (SGW)} is the first data tunnel anchor point
that UE sessions connect through (GTP tunnels). The \textbf{Packet
  Data Network Gateway (PGW)} act as the egress point for a large
number of UE data bearers (GTP tunnels); they fan out to multiple
SGWs.  \textbf{Evolved NodeB (eNodeB)} devices are the wireless access
points of the 4G EPS. They bridge the radio access network (RAN)
through which the mobile endpoints (UEs) directly communicate with the
evolved packet core (EPC). GTP tunnels are established for each UE
between the eNodeB it is associated with and an upstream SGW.  The
eNodeB also initiates session setup and default data bearer
establishment when UEs attach, acting as a proxy for UE to MME
control plane signalling. eNodeBs covering adjacent cells coordinate
through the MME and possibly with one another to accomplish handover
as endpoints move.

\subsection{The Vehicle Environment}

\textbf{Discuss some of the important aspects of vehicles, suchs as sensors and such, here.}
