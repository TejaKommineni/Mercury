\subsection{Related Work}

There has been plenty of attention paid to effecient and reliable
delivery of messages within VANETs.  Much of this focuses on multi-hop
clustering and hybrid use of evolved packet system RAN (LTE).  The
VMaSC~\cite{ucar2016multihop}, MDMAC~\cite{wolny2008modified}, and
NHop~\cite{zhang2011novel} systems attempt to form stable mobile
802.11p clusters, using the LTE network to bridge between disconnected
clusters. These solutions largely ignore the details of the mobile
core network, glossing over questions of component placement and the
resulting effects on latency. Moreover, these systems are
complimentary to \name~in that they can be used to reduce LTE
resource contention and improve reliable transfer of messages. Such
integration would however come at the cost of additional complexity,
failure modes, and latency due to additional network path segments.

From the publish-subscribe perspective, there is a large volume of
prior work on traditional pubsub mechanisms~\cite{ps1,ps2,ps3,ps4}.
This work is complementary to ours since it focuses on the useful
aspects of pubsub, which we largely wish to reuse. There has also been
work on pubsub systems focused on mobile endpoints. The
MoPS~\cite{nasim2014mobile} publish-subscribe system scales
efficiently for large numbers of clients and deals well with changing
broker association.  \name~could replace the \pubsub system used in
the prototype with MoPS to better target mobile endpoints. However,
MoPS does not include the area of interest concept, which would
continue to be handled by the \name~broker. A paper by
Pongthawornkamol et al~\cite{pongthawornkamol2007analysis} looks at
pubsub in the context of ad hoc wireless networks. However, this work
only performs simulations of mechanisms, and does not consider a
larger top-down vantage point (important for coordination in a large
ITS).

Location-aware messaging, or geo-routing, in vehicular networks has
been studied fairly extensively~\cite{bilal2013position}.
Nevertheless, we find that most work has only proposed and simulated
mechanisms. Furthermore, many studies look at ad hoc networks and
fine-grained positioning of endpoints within vehicle clusters.  While
such mechanisms may be helpful for real-time collision avoidance, we
argue that they tend to be overly complex and unnecessarily constrain
the communication domain to clusters of endpoints versus a central
system with a global view. The argument against centralized systems is
frequently that mobile networks are overloaded. This may be true in
instances of particularly high device concentration (e.g. music
festivals), but we found no studies showing a general lack of
available RAN resources outside of such crowded contexts.
Furthermore, in an ITS environment, vehicle populations and
anticipated growth could be used to properly size capacity.

The related work outline above focuses on particular aspects of
messaging.  To the best of our knowledge, no messaging system targeted
at mobile endpoints simultaneously incorporates aspects we consider
crucial to a holistic messaging platform for Intelligent
Transportation Systems.  Such a system should enable a global view of
endpoints to facilitate coordinated decision-making with complete
data. It should take advantage of mobile network environment
mechanisms (e.g. eMBMS) to reduce overhead and latency. It should
consider the placement of components within the mobile network and the
impact of this placement.  Finally, an ITS messaging system should
provide a location-aware addressing mechanism. The \name~messaging
system is designed with all of these aspects in mind.
