\section{Evaluation}

\begin{itemize}
\item Discuss evaluation approach.
  \begin{itemize}
  \item Do evaluation in PhantomNet using OpenEPC with emulated RAN.
  \item Use vehicle mobility model, SUMO, to drive realistic mobility scenarios.
  \item Use SUMO output (position, primarily), to trigger handover.
  \end{itemize}
\item Types of evaluations to perform.
  \begin{itemize}
  \item Functional:
  %%% Functional Evaluation.
  \begin{figure}[ht]
   \begin{center}
    \includegraphics[width=0.5\textwidth]{figs/simulated.png}
    \caption{Mercury event simulation.}
    \label{fig:simulated}
   \end{center}
  \end{figure}
   %%% Functional Evaluation.
  \begin{figure}[ht]
   \begin{center}
    \includegraphics[width=0.5\textwidth]{figs/aoi.png}
    \caption{Mercury Area Of Interest.}
    \label{fig:aoi}
   \end{center}
  \end{figure}
  In order to verify Mercury we have written a Message Simulator that mimics
  real time vehicular movement. This simulator provides us with two 
  functionalities. Firstly it helps us in specifying the number of vehicles to
  be instantiated within the system, their location and the speeds associated 
  with them. Secondly we can use the simulator to mock an event at any given 
  location. All the vehicles that are within the region of this mocked event 
  publish a message to the Mercury Adapter about they sensing an event. 
  
  Using the Message Simulator we have verified the functionality of the system.
  To do this we have initially set up ten vehicles in the simulated ecosystem.
  Then mocked an event collision at location (5,2) with a radius of 2 as in 
  Fig3. All the vehicles within this region have sensed the event and published 
  a message to the Mercury. We have taken care that the number of vehicles that
  sense the event and publish the messages to Message Adapter will be greater
  than the threshold needed for the Message Broker to trigger an Area of 
  Interest message. The AOI message from Message Broker is represented as 
  bigger circle in the Fig4.

  By the time AOI message from Message Broker has reached the vehicles we have
  observed the vehicle which has sensed the event at location (6,2) has moved
  away from the region. This vehicle didn’t receive any alert from Mercury. 
  There were two other vehicles that are at location (3,5) and (8,6) and have
  moved into the AOI calculated by Message Broker both these vehicles received
  the message. And the three of four vehicles which have sensed the event were
  still falling within the Message Broker’s AOI and have received an alert. 
   
  By this experiment we have verified that all the vehicles within the AOI of
  Message Broker have received an alert irrespective of sensing the event and
  those vehicles which are out of the AOI have not received the vehicle.

  
  \item Functional: Endpoints connect and are tracked properly (handover).
  \item Functional: Areas of Interest are interpreted properly.
  \item Scaling: Run simulated/emulated scenarios with dozens of endpoints.
  \item Scaling: Increase message/sec load and observe system response times.
  \item Trust: Try to forge and alter messages (MITM)
  \item Trust: Try to manipulate system state (``i.e., game the system'').
  \end{itemize}
\end{itemize}
