\section{Introduction}

Intelligent Transportation Systems (ITS)~\cite{zhang2011data} is an emerging
area that includes many ideas and mechanisms for improving the safety,
quality of experience, and communication capabilities of
commuters. One particularly important aspect of ITS is
vehicle-to-vehicle and infrastructure-to-vehicle
communication. Vehicle area networks (VANETs)~\cite{hartenstein2008tutorial} have been
introduced to meet the challenges of mobile network actors with
rapidly changing associations arising from dynamic proximity to
infrastructure and other vehicles. Together with mobile networking a
la the 3GPP evolved packet system~\cite{4G}, robust communication
approaches involving hybrid LTE and 802.11p~\cite{802.11p} with
dynamic multi-hop clustering have been
proposed~\cite{ucar2016multihop,wolny2008modified,zhang2011novel}.
Considering safety and
other types of information exchange in an ITS, applications and events
have been identified that should be supported~\cite{camp2005vehicle,papadimitratos2009vehicular}, along
with constraints such as latency for delivery and scope
(radius). Publish-subscribe systems provide a means to efficiently
distribute messages and events between infrastructure and mobile
actors in an ITS. Such pubsub systems can support peer-to-peer and
infrastructure-sourced messages at scale~\cite{nasim2014mobile}.  While
research has been done in the areas of VANET communication and
mobility-aware publish/subscribe systems, holistic compositions of
these technologies is lacking.

Transportation safety issues are also considerable.  The total number
of vehicle sales in the United States averaged 15.43 million from 1993
up through 2015. The year 2015 saw a surge in sales, which climbed to
17.76 million vehicles and is expected to rise with the improving
economy~\cite{te}. Increasing vehicle populations will likely result
in increasing accident rates. A total of 32,675 people died in motor
vehicle crashes in 2014~\cite{iihs}. We argue that technology advances
should be used to mitigate increasing causualties and ensure safe
journey. A key focus in this paper is using state-of-art
5G~\cite{5gvision} networks to capture vehicle telemetry and pass
along safety events to vehicles that are revelvant to their
surroundings.

The setting in which any transportation messaging system operates
requires reliable, fast paced communication. However, the latency
incurred by present solutions is not adequate for time-sensitive
interactions (e.g., for real-time inter-vehicle
notifications). Although there are architectures~\cite{mh} that use
the 802.11p communication paradigm for low latency, they lack good
trust mechanisms. Such ad-hoc networks are also often unstable,
requiring frequent communication path adjustments that result in
losses and overhead. 3GPP-style mobile networks provide advantages in
stability, reachability, and security.

This paper introduces \name; a integration of pubsub systems and
client endpoint mechanisms for low-latency message transport.  Part of
the holistic vision that \name espouses is mobility-specific
geographic areas of interest (AOI). These areas map to slices of
physical locality that are relevant to particular events.  For
example, a traffic accident and resulting congestion are relevant to
vehicles en route to the accident location, back past potential
egresses to alternate routes.  \name takes into account
location-specific context to calculate the relevant dynamic set of
vehicles for message transmission.

This paper makes the following contributions:
\begin{itemize}

\item A holistic end-to-end messaging system design.

The design of \name highlights the mechanisms by which
publish/subscribe systems and mobile endpoints can be realized in 4G
or 5G mobile network systems. Broker aggregation services communicate
through Adapters to Client Endpoints at the edge. \name's design
accommodates flexible deployment scenarios.

\item A prototype implementation and evaluation of the \name ITS
  messaging system.

We implemented a prototype of the \name components and evaluated it
on the PhantomNet~\cite{banerjee2015phantomnet} testbed.  This
prototype makes use of OpenEPC~\cite{corici2010openepc} version 5,
including its simplified emulated radio access network (RAN)
environmnet. We drive the evaluation of \name with our own mobility
simulator, and report on the former's functional and latency
characteristics.

\end{itemize}

The remainder of this paper is organized as follows: Section 2 covers
background, including putting \name into context within related
work, introducing mobile networking environment concepts, and
describing the vehicular endpoing context. Section 3 provides an
overview of the design goals and components of \name, while section
4 delves into the details of the design. Section 5 looks at how
\name can be positioned within different mobile networking
deployments.  Section 6 briefly discusses the \name prototype
implementation. We cover our vehicle simulator and provide an
evaluation of the prototype in section 7. Section 8 discusses
observations resulting from the design and implementation of \name,
and where we think it should go next.  Finally, section 9 concludes.
