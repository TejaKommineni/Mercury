\section*{Introduction}

Intelligent Transportation Systems (ITS)~\cite{ITS} is an emerging
area that includes many ideas and mechanisms for improving the safety,
quality of experience, and communication capabilities of
commuters. One particularly important aspect of ITS is
vehicle-to-vehicle and infrastructure-to-vehicle
communication. Vehicle area networks (VANETs)~\cite{VANET} have been
introduced to meet the challenges of mobile network actors with
rapidly changing associations arising from dynamic proximity to
infrastructure and other vehicles. Together with mobile networking a
la the 3GPP evolved packet system~\cite{3GPP}, robust communication
approaches involving hybrid LTE and 802.11p~\cite{802.11p} with
dynamic multi-hop clustering have been
proposed~\cite{ucar2016multihop}~\cite{wolny2008modified}~\cite{zhang2011novel}.
Considering safety and
other types of information exchange in an ITS, applications and events
have been identified that should be supported~\cite{vanet-apps}, along
with constraints such as latency for delivery and scope
(radius). Publish-subscribe systems provide a means to efficiently
distribute messages and events between infrastructure and mobile
actors in an ITS. Such pubsub systems can support peer-to-peer and
infrastructure-sourced messages at scale~\cite{nasim2014mobile}.  While
research has been done in the areas of VANET communication and
mobility-aware publish/subscribe systems, holistic compositions of
these technologies is lacking.  This paper introduces Mercury; a
5G-based~\cite{5G} integration of pubsub systems and VANETs for rapid
and reliable message transport.  Part of the holistic vision that
Mercury espouses is mobility-specific geographic areas of interest
(AOI). These areas map to slices of physical locality that are
relevant to particular events.  For example, a traffic accident and
resulting congestion are relevant to vehicles en route to the accident
location, back past potential egresses to alternate routes.  Mercury
takes into account such relationships to calculate the relevant dynamic
set of vehicles for message transmission.

This paper makes the following contributions:
\begin{itemize}

\item Integration of publish/subscribe systems with VANETs using 5G mechanisms.

The design of Mercury highlights the mechanisms by which
publish/subscribe systems and VANETs can be combined in a 5G
ecosystem. Message interception toward the pubsub broker, and
injection toward the mobile vehicles is accomplished using SDN
techniques over control and data paths in the EPS. Injected messages
make use of multicast slots in the LTE radio access network (RAN) for
efficient dissemination.  We also show how concepts such as
CloudRAN~\cite{checko2015cloud} and mobile edge computing~\cite{mobile-edge}
can further enable our system to operate within tighter latency bounds.

\item Introduction of practical methods for determining areas of interest.

A road database is annotated with ingress and egress points, and
boundaries for vehicle route placement.  Vehicles report their
positions as part of the telemetery collected by the message
distribution broker.  Together with context-specific details for event
types (accident, obstruction in road, lane closure, emergency vehicle
approaching, etc.), regions are dynamically computed.

\item A design and prototype implementation, and evaluation of the
  Mercury ITS messaging system.

We implemented a prototype of Mercury and evaluated it on the
PhantomNet~\cite{banerjee2015phantomnet} testbed.  This prototype makes use of
OpenEPC~\cite{corici2010openepc} in the core network and Open Air Interface
(OAI)~\cite{OAI} at the edge (RAN).  We drive the evaluation of Mercury
using an adaptation of the SUMO~\cite{behrisch2011sumo} mobility model, and
report on the former's scaling and latency characteristics.

\end{itemize}

The remainder of this paper is organized as follows: Section 2 covers
the design of the Mercury message handling system.  Section 3
discusses the implementation of Mercury, including integration of the
MoPS pubsub system and \emph{TAP}, our mechanism for interposing on
EPS mobile control signalling and data flows for message delivery.
Section 4 discusses our experimentation setup in Phantomnet and
section 5 presents the results obtained.  Section 6 covers related
work, and section 7 concludes.
