\section{Design Details}
\label{sec:design-details}

Each \name~component runs as a self-contained application. The
Broker and Adapter(s) communicate over unmodified \pubsub. Client
Endpoints communicate with the Adapter over UDP. Client-side
applications using \name~communicate over the local loopback with
the Client Endpoint process. Service Endpoints use \pubsub to
communicate with the Broker and talk to Clients via
Adapter(s).

\subsection{The messages and events}

The \name~Broker we have implemented handles different types of events. 
Emergency Event, This is an event that is published by a vehicle when it senses
an emergency situation. Any message indicating such an event is immediately 
processed by the \name~broker and sends back an area of interest (AOI)
that has to be alerted for this event.
Moving Objects Event, This event indicates that there is an object that is 
crossing a road. For such events, whenever a predicate calculated by scheduler 
evaluates to true we inform the vehicles in the AOI calculated.
There are other types of events such as Collision, Obstacle, Congestion and 
Blocked. All, these events are handled similarly but they differ in two things.
One is the predicate function and the second is the frequency at which the 
schedulers run. Our system also supports an other type of event called 
Area Of Interest. This is different from the aoi which is calculated by
\name~broker. In this type of event the user is interested in knowing about 
the different events happening at a particular location. Whenever an event of 
type area Of interest is received by the \name~broker. It interacts with 
different handlers and determines the bin for each event type in which the 
requested aoi falls into. Then runs a predicate on these bins and learns about 
the traffic conditions in that area which is communicated back to the user.

\begin{table*}[ht]
  \centering
  \begin{tabular}{| l | l |}
    \hline
    \textbf{Field} & \textbf{Description} \\ \hline \hline
    UUID & Unique identifier for message. \\ \hline
    Session ID & Identifies session between Client and Adapter. \\ \hline
    Message Type & Distinguishes intent (to/from Broker, PubSub, etc.) \\ \hline
    Source Address & Client ID, Service ID, Broker, or Adapter. \\ \hline
    Dest Address & Client ID, Service ID, Broker, Adapter, or AOI. \\ \hline
    Topic & Indicates topic message goes to or came from (PubSub only). \\ \hline
    KeyVal & Generic key-value \emph{pairs}. Telemetry and other data is encoded here. \\
    \hline
  \end{tabular}
  \label{tab:mesgformat}
  \caption{Message fields found inside individual \name~messages.}
\end{table*}

Messages must fit into a single UDP packet (maximum of 64K).  Each is
self contained, with required identifiers, addresses, and message
payload.  There are two top-level types of messages in the \name~system: 
session and pubsub.  Session messages are transmitted between
Client Endpoints and associated Adapters.  Pubsub messages can
essentially be exchanged between any \name~components.
Table~\ref{tab:mesgformat} shows the fields present in \name~Session
and PubSub messages. Each \name~Address contains a destination and
source address.  The source address can be an Adapter, a specific
Client, a Service, or the Broker.  The destination address can be any
of those listed for source, and additionally may be a broadcast (all
clients, or Area of Interest geo-address).  Each message contains
key/value pairs in the payload which are interpreted by specific
applications. For example, emergency alerts published by the Broker
may be consumed by an alert display application at the Client
Endpoint.  Client Report message contain mobile telementry
information: GPS location, direction, and speed. These reports have
semantic meaning to \name, and so are tracked by the Client
Endpoint, Adapter, and Broker components.  In our prototype
implementation, the Broker has had emergency event message handling
folded into it's logic.  Therefore, the prototype Broker also
interprets safety messages sent via pubsub topics such as
``Collision'' and ``Object\_Hazard''. We envision that such message
handling would ultimately be moved to an external Service Endpoint
application.

\subsection{Areas of Interest}
\label{sec:aoi}

\name~uses Area of Interest geo-addressing to target specific sets
of Client Endpoints.  When the Broker wishes to send a message that is
specific to a location context (e.g., a warning triggered from
multiple collision reports), it embeds the desired geo-address as the
destination, and sends this along with the message toward the
Clients. This message is picked up from the pubsub by the Adapters.
These check their coverage area and forward the message along to
Clients within the defined geo-address that they serve (or drop if
there is no overlap).  When an AOI-addressed message is received by a
Client Endpoint, it checks its current location to determine if the
message applies, and processes if so.

\subsection{\name~broker}
  
\name~Broker handles events in two steps: In the first step events
published are passed on to the respective handlers where we have a
filter that determines the different locations from which the feeds
are coming in for that specific event.This will give us the messages
published at different locations for an event.Each location can be
considered as a bin. We maintain for each bin the count of messages
published, the center point and the radius for that bin.  Whenever a
message comes into a bin it carries along the coordinates from which
this event has been published. We apply a simple mean with the
existing center point to determine the new center point. Thus, at
every instance we keep updating the center point and radius.In the
second step we have schedulers for each event that get triggered at
equal intervals based on the event type.  These schedulers empty the
bins of their respective event and run a predicate on them which is a
simple Boolean formula that checks for a condition.  A predicate is
usually defined over some aggregation function expressed on messages;
when the predicate evaluates to true, the \name~broker publishes an
event to the system with the center point and the radius which we have
calculated for that bin. These are used to determine the area of
interest(AOI), AOI is the region that is determined by the \name~broker
and constitutes all the clients that have to be alerted about a
situation.

\subsection{Pubsub service}

Within \name~we require a publish-subscribe system for communication
between different entities.  This requirement directly arises from the
need for each vehicle to send and listen to particular types of
messages. Pubsub systems are well-suited for these activities as we
can publish and subscribe to different events. In our system
components such as Client Endpoints and the Message Broker publish and
subscribe to events.  Given this communication pattern, we developed
the set of requirments we need in a publish-subscribe mechanism and
surveyed available options.

As mentioned in section~\ref{sec:design}, we selected Apache Kafka as our pubsub
component. It is a distributed streaming platform which lets users to publish 
and subscribe to streams of records. Kafka is run as a cluster on one or more 
servers. The Kafka cluster stores streams of records in categories called topics.
Each record consists of a key, a value, and a timestamp. A topic is a category 
or feed name to which records are published. In our case it is the event type.
Topics in Kafka can be multi-subscribed; that is, a topic can have one or more 
consumers that subscribe to the data written to it. In our implementation
all the values written to a topic are listened by Message Broker. But, we could
extend this to different service end points listening on the same topic.

For each topic, the Kafka cluster maintains a partitioned log. Each partition 
is an ordered, immutable sequence of records that is continually appended to a 
structured commit log. The records in the partitions are each assigned a 
sequential id number called the offset that uniquely identifies each record 
within the partition. In our system we create a single partition for each topic.
As within a partition order is maintained we can traverse through records using
the offset of the record. Each partition is replicated across a configurable
number of servers for fault tolerance. The Kafka cluster retains all published 
records whether or not they have been consumed—using a configurable retention 
period.

\subsection{Mobile network messaging adapter}

The \name~Adapter follows a multi-threaded, event-driven execution
pattern. It is highly modular, with clear separation between client
state tracking, network-specific address mapping, and transport
functionality.  This allows new modes of core network communication to
be added alongside (or in place of) existing mechanisms.  Different
threads of execution handle pubsub, UDP communication, and scheduled
tasks (e.g., session checking); all are coordinated through the main
thread.

\name~is designed so that any number of Adapters can connect to the
\pubsub system. Each will send client telemetry reports and pubsub
messages along. A single broker message sent via pubsub reaches all
connected Adapters, which filter as necessary. For example, an Adapter
will drop messages with Area of Interest destinations that do not
overlap with its coverage area. These coverage areas are configured by
the operator.
  
\textbf{Endpoint identification/tracking} The Adapter implements
client session tracking. When a client comes online, it goes through a
lightweight handshake with the Adapter. An INIT message is sent to the
adapter, which immediately responds with an ACKNOWLEDGE message. The
client includes a telemetry report in its INIT message so that the
Adapter and Broker have immediate knowledge of its current state. Each
client is provided with a session ID token that must be used in
subsequent messages.  The Adapter sends out periodic HEARTBEAT
messages to clients so that they can measure liveness in the absence
of other messages.

The Adapter maintains a small amount of state for each client. This
includes its unique ID, current mobile telemetry (no history), session
ID, and simple statistics (timestamp for last message received,
message count). The Adapter also maintains a mapping from the client's
ID to its mobile core network address. Changes in this address are
automatically detected and the mapping updated. The Adapter treats
periodic client reports as heartbeats, and resets corresponding
individual timers as these are received. Sessions are pruned after not
receiving a report for three times the reporting interval, or after
receiving an explicit CLOSE message from the client.

\textbf{Unicast UDP transport} In the prototype implementation, we use
UDP unicast packets for all communication between Adapter and Client
Endpoints. Ideally we would use eMBMS broadcasts for AOI and broadcast
(e.g. heartbeat) messages, and we plan to do so in future work. Each
UDP packet is a self-contained \name~message. The contents are
encoded using Google's Protocol Buffers~\cite{buffers2011google} for
performance and space efficiency. For AOI destination addresses, the
Adapter calculates the set of clients to transmit a message to and
unicasts to each individually. For broadcasts, it transmits to all
clients with valid sessions.

\textbf{Pubsub passthrough} Messages from established clients that are
intended for the pubsub system are directly published to the indicated
topic by the Adapter. Likewise, messages received from the Broker via
pubsub are sent to specific endpoints or broadcast to all clients
handled by the Adapter (as appropriate).
    
\subsection{Client Endpoint mechanisms}

The \name~Client Endpoint connects to the Adapter over UDP. It
establishes a session using a simple handshake; an INIT message is
sent to the adapter, which it replies to with an acknowledgement. The
client will continue trying to connect until it receives the
acknowledgement, backing off linearly with random jitter to prevent
synchronization with other clients (prevent in-cast overload). The
client will also try to reestablish its session if it does not receive
a heartbeat from the Adapter after three heartbeat intervals (tunable).
Messages are encoded using Google's Protocol Buffers.

The client gathers movement telemetry via the local sensors of the
device it is running on and sends these along periodically as report
messages to the Adapter.  As mentioned, the Adapter uses these reports
to measure the liveness of the connected clients.  Some random jitter
is added initially to the report interval to again avoid
synchronization with other clients.

The client listens over the loopback interface for connections from
local applications.  Similar to what the Adapter does for its Client
Endpoints, a Client Endpoint relays subscription requests and messages
from applications through to the pubsub/broker.  Protocol Buffers are
also used between clients and applications for encoding the messages.

%%%%%%%%%%%%%%%%%%%%%%%%%%%%%%%%%%%%%%%%%%%%%%%%%%%%%%%%%%%%%%%%%%%%%%%%%%%%%%
% 
% We may not have time/space to include any goal alignment text.
%

\comment{
\subsection{Design Goal Alignment}

The remainder of the design section of the paper is dedicated to a
deeper exploration of the design of \name~and how it aligns with the
afforementioned goals.

\subsubsection{Delivering Relevant Content}

\begin{itemize}
\item Discuss what makes content relevant, and to whom it is relevent.
\item Metric(s) for relevancy?
\item Introduce and expand on Areas of Interest (dynamic grouping).
\item Motivate and describe publish/subscribe mechanism.
\end{itemize}

\subsubsection{Reliable Communication}

\begin{itemize}
\item Differing message requirements, based on type
\item Effects of latency, minimization through proximity to network edge
\item Delivery service levels (guaranteed, best effort)
\item Robust transit mechanisms?
\end{itemize}

\subsubsection{Deployment Scenarios}

\begin{itemize}
\item Components/service designed to integrate into different deployments
\item Minimal cost/effort: pubsub/broker centrally located with PGWs
\item Incrementally closer: MTSO or metro-area (CloudRAN) deployment
\item At the edge: Deployed as a service running on eNodeBs
\end{itemize}

\subsubsection{Integrating Trust}

\begin{itemize}
\item PKI to prevent identity forgery
\item Message integrity codes to prevent tampering
\item Sequence numbers to prevent replay attacks
\item Centralized vetting by broker to detect bad behaviors/actors
\item Encryption for privacy? We should give this some thought.
\end{itemize}

\subsubsection{Technology Reuse and Integration}

\begin{itemize}
\item Composing from a good collection of parts
\item Existing pubsub, PKI, MAC/MIC, eMBMS mechanisms, SMORE
\item Working around limitations
\item Transparent interposition
\item Open possibilities in the 5G landscape
\end{itemize}
}
