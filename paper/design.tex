\section*{Design}

Delivering messages, critical and casual, to mobile users and
endpoints that are interested in them is the overarching premise of
this work. Our vision for realizing an end-to-end mobile message
delivery service design principles revolve around {\bf X} central
goals:

\begin{itemize}
\item Relevant content

The service should provide mechanisms to target groups of endpoints
which have explicit or implied interest in messages. A message may be
important because an end user specifically asked for the content based
on its attributes (nearby gas prices). Alternatively, a message may be
deemed relevant for the endpoint because it is related to an emergent
event (vehicle accident ahead).

\item Robust, low overhead, low latency communication

Message intent drives content delivery requirements.  For example,
different types of emergency service messages have been identified for
VANETs, each having distinct latency
requirements~\cite{vanet-msg-reqs}. The service should strive to
minimize latency to provide on-time delivery with headroom for outlier
delays. Messages should be categoriezed according to their relative
importance and processed accordingly (e.g., emergency info before
consumer content).

\item Flexible deployment

Adoption of the service is bolstered by adaptability to different
mobile networking environments.  Such accomodation allows for
deployment into LTE networks with different geographic EPC service
placements and degrees of maleability.  The service should allow for
centralized metropolitan area integration (CloudRAN) and distributed
edge deployment (peer-to-peer eNodeB).

\item Message trust and integrity

Messaging is vulnerable to various attacks, particularly if peers are
used as transits such as in 802.11p multihop clusters. A combination
of PKI, message integrity checks, and centralized vetting should be
employed to curtail abuse and instill trust.

\item Reuse of effective technologies

It is our contention that an end-to-end service should not supplant
existing mechanisms useful for achieveing its composition.  Indeed, it
is counterproductive to introduce new service components that induce
unnecessary changes and capital investments. The messaging service
should strive to work alongside existing mobile network protocols and
services.  Only where existing mechanisms do not provide key
functionality or do not give adequate service levels should changes be
introduced.  Such changes should be as minimal and transparent as
possible to foster compatibility and ease of adoption. On the other
hand, considering less constrained future mobile network
architectures~\cite{5G}~\cite{MobilityFirst} is equally important.

\end{itemize}

\subsection{Mercury Architecture Overview}

{\bf INSERT MERCURY ARCHITECTURE PICTURE}

\begin{itemize}
\item Talk about mobile networking ecosystem and where Mercury fits in.
\item Scope of work.
\item Introduce components of Mercury: brief descriptions and relationships
\end{itemize}
  
The remainder of this section of the paper is dedicated to a deeper
exploration of the design of Mercury and how it aligns with the
afforementioned goals.

\subsection{Delivering Relevant Content}

\begin{itemize}
\item Discuss what makes content relevant, and to whom it is relevent.
\item Metric(s) for relevancy?
\item Introduce and expand on Areas of Interest (dynamic grouping).
\item Motivate and describe publish/subscribe mechanism.
\end{itemize}

\subsection{Reliable Communication}

\begin{itemize}
\item Differing message requirements, based on type
\item Effects of latency, minimization through proximity to network edge
\item Delivery service levels (guaranteed, best effort)
\item Robust transit mechanisms?
\end{itemize}

\subsection{Deployment Scenarios}

\begin{itemize}
\item Components/service designed to integrate into different deployments
\item Minimal cost/effort: pubsub/broker centrally located with PGWs
\item Incrementally closer: MTSO or metro-area (CloudRAN) deployment
\item At the edge: Deployed as a service running on eNodeBs
\end{itemize}

\subsection{Integrating Trust}

\begin{itemize}
\item PKI to prevent identity forgery
\item Message integrity codes to prevent tampering
\item Sequence numbers to prevent replay attacks
\item Centralized vetting by broker to detect bad behaviors/actors
\item Encryption for privacy? We should give this some thought.
\end{itemize}

\subsection{Technology Reuse and Integration}

\begin{itemize}
\item Composing from a good collection of parts
\item Existing pubsub, PKI, MAC/MIC, eMBMS mechanisms, SMORE
\item Working around limitations
\item Transparent interposition
\item Open possibilities in the 5G landscape
\end{itemize}
