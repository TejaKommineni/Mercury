\section{Discussion}

The design of \name, the possible deployment scenarios it can
accommodate and the evaluation of prototype implementation of \name{} 
clearly suggest that \name{} is a fit for both present and future
Intelligent Transportation Systems. Yet, there are few areas where we
can further improve to make \name{} more appealing.

At present, \name{} uses AOI as a main primitive in delivering
messages to the clients. \name{} Broker determines the AOI and this is
processed by the Message Adapter to deliver messages to respective Clients.
The AOI that we are implementing or envisioned in this paper is based
on the circle centered at a point with a given radius. Clearly, when
we are using the circular region as metrics there could be vehicles
which are not affected by the event yet receive the alert message from
\name{} because they fall inside this region. This is where we think
AOI calculation should also add another dimension to its measurement,
namely direction. This comes from the basic concept that an event
happening on one side of the road may or may not affect the vehicles
moving on the other side of the road. Looking at the directions we can
also detect collisions between vehicles using the reports they send to
the system.  Hence, this is an interesting future work area.

Presently, the Message Adapter doesn't implement any significant
security measures when dealing with the messages that are moving in
and out of the system. We only check if the sender of the message has
a session with \name{} and then forward it to the \name{} Broker. The
verification that is done here can be further supplemented by other
methods such as self-certifying endpoint IDs and signed messages (to
prevent spoofing), and data analysis at the Broker looking for
anomalies (to prevent gaming).  Without such protections, miscreants
could exploit and/or spoof messages without sensing an event or in any
other anticipated way that could generate profits to them or cause
harm to others.

The Message Broker receives all sensed events from the Message Adapter
and performs computation on them to determine AOI. Here, we can see
that Message Broker depends on the incoming messages to figure out
whether an incident/event is happening. It doesn't have any other way
to know about an event except for the incoming messages. But, if we
have a close look at \name{} we also have another source of
information; the reports that are periodically sent by vehicles.  We
can make the Message Broker intelligent enough to look at these
reports and detect an event. It could further decrease the latency in
the system and we could make more informed decisions a step before a
threshold of vehicles get to sense this event and report it to the
system. For example, all vehicles which are in a region experiencing
congestion move at a slow speed. Looking at the reports from vehicles,
we could detect the slowing down of vehicles in any region and send a
congestion alert to relevant vehicles (via AOI).

Section~\ref{sec:deployments} mentions a handful of different
deployment scenarios. We would like to extend \name{} into these
different scenarios and measure its performance. In particular,
we would like to make use of eMBMS for broadcasting AOI and
heartbeat messages to Clients.  We would also like to explore
the edge deployment scenario. This scenario has the greatest
potential for minimizing latency, and also has interesting fault
tolerance properties and challenges. Keeping a mesh of agents
running at each eNodeB bypasses issues in the core network, but
means manging message routes and maintaining healthy paths.

Finally, there are a number of things we can do to improve the
performance of \name. We could increase the performance by increasing
parallelism in the system such as scaling Message Adapter and Message
Broker. We could also take advantage of the consumer groups concept in
the Apache Kafka that help us in instantiating multiple consumers on a
single topic. This way all the messages that come to a single topic
can also be processed by multiple consumer instances giving us high
throughput. We can also take advantage of the retention period in the
Kafka to go back in time and receive any alert if missed. This could
probably be an interesting option to look at because vehicles may tend
to lose connection to intermittently and when they establish
connection again with the system they might be interested in listening
those events that it missed.

\comment{
\begin{itemize}
\item Lessons/insights extracted from design/implementation/evaluation.
\item Future work.
\item Limitations of approach.
\end{itemize}
}
