\section{Discussion}

The design of \name, the possible deployment scenarios it can
accommodate and the evaluation of prototype implementation of \name~
clearly suggest that \name~is a fit for both present and future
Intelligent Transportation Systems. Yet, there are few areas where we
can further improve to make \name~more appealing. Below we discuss
the areas that can be further focused upon:

At present, \name~uses AOI as a main primitive in delivering
messages to the clients. \name~Broker determines the AOI and this is
processed by Message Adapter to deliver messages to respective UEs.
The AOI that we are implementing or envisioned in this paper is based
on the circle centered at a point with a given radius. Clearly, when
we are using the circular region as metrics there could be vehicles
which are not affected by the event yet receive the alert message from
\name~because of falling under this region. This is where we think
AOI calculation should also add another dimension to its measurement
namely direction. This comes from the basic concept that an event
happening on one side of the road may or may not affect the vehicles
moving on the other side of the road. Looking at the directions we can
also detect collisions between vehicles using the reports they send to
the system.  Hence, this is an interesting future work area.

Presently, the Message Adapter doesn't implement major security
measures when dealing with the messages that are moving in and out of
system. We only check if the sender of the message has a session with
\name~and then forward it to the \name~Broker. The verification
that is done here can be further supplemented by other methods such
as.. else, the miscreants could try to exploit this vulnerability by
spoofing messages without sensing an event or in any other anticipated
way that could generate profits to them or cause harm to others.

Message Broker receives all the sensed events from Message Adapter and
performs computation on them to determine AOI. Here, we can see that
Message Broker depends on the incoming messages to figure out whether
an incident/ event is happening. It doesn't have any other way to know
about an event except for the incoming messages. But, if we have at
close look at \name~we also have another source of information that
is the reports that are periodically sent by vehicles to the
\name. If we can make the Message Broker intelligent enough to look
at these reports and detect an event. It could further decrease the
latency in the system and we could make more informed decisions a step
before a threshold of vehicles get to sense this event and report it
to the system. For example, all those vehicles which are in a region
experiencing congestion move at a slow pace. Looking at the reports of
the vehicles if we could detect the slowing down of vehicles in any
region we can calculate AOI and send it to those vehicles without
waiting for sensed messages from vehicles.

Also, as mentioned in the design section about the possible different
deployment scenarios. We could try to implement \name~in these
different scenarios and measure its performance. We could increase the
performance by increasing parallelism in the system such as scaling
Message Adapter and Message Broker. We could also take advantage of
the consumer groups concept in the Apache Kafka that help us in
instantiating multiple consumers on a single topic. This way all the
messages that come to a single topic can also be processed by multiple
consumer instances giving us high throughput. We can also take
advantage of the retention period in the Kafka to go back in time and
receive any alert if missed. This could probably be an interesting
option to look at because vehicles may tend to lose connection to
intermittently and when they establish connection again with the
system they might be interested in listening those events that it
missed.

\comment{
\begin{itemize}
\item Lessons/insights extracted from design/implementation/evaluation.
\item Future work.
\item Limitations of approach.
\end{itemize}
}
