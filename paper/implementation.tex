\section{Implementation}

We next briefly cover some of the essential implementation details of
the \name prototype. Section~\ref{sec:design-details} discusses our
selection of Kafka for the pubsub component. Google's Protocol Buffers
were used for efficient on-the-wire binary encoding of messages. Such
encoding is used on all non-pubsub communication paths. The \name
prototype was written in python in about 2,200 lines of code. We chose
Python because it was easy to rapidly produce a working system in this
high-level langauge, and because it has good support for parallel
processing. Integrations exist for Python for the other technologies
used. Concurrent threads are employed to capture and buffer incoming
messages from both the pubsub and UDP point-to-point channels. An
internal scheduler was implemented to perform periodic tasks, and
supports randomized offsets.

The prototype implementation targets the centralized PGW deployement
scenario where the core \name components (Adapter, Broker, and
pubsub) are positioned immediately on the other side of a standard
3GPP data bearer egress point. This deployement scenario affords low
latency overhead when EPC core components are located in a datacenter
near the network edge.  We covered this deployement scenario in
section~\ref{sec:metro-deploy}.  Note that we did not take advantage
of any specific aspects of the mobile environment in our prototype,
which is left as future work.
